\documentclass[11pt,a4paper]{article}

\usepackage[utf8]{inputenc}
\usepackage[T1]{fontenc}
\usepackage{lmodern}
\usepackage{geometry}
\geometry{margin=1in}
\usepackage{microtype}
\usepackage{amsmath,amssymb,amsfonts,amsthm}
\usepackage{mathtools}
\usepackage{physics}
\usepackage{siunitx}
\usepackage{graphicx}
\usepackage{booktabs}
\usepackage{hyperref}
\usepackage[nameinlink]{cleveref}
\usepackage{authblk}
\usepackage{enumitem}
\usepackage{tikz}
\setlist{nosep}

\hypersetup{
  colorlinks=true,
  linkcolor=blue!60!black,
  citecolor=blue!60!black,
  urlcolor=blue!60!black
}

\numberwithin{equation}{section}

% Macros
\newcommand{\R}{\mathbb{R}}
\newcommand{\EE}{\mathbb{E}}
\newcommand{\KL}{\operatorname{KL}}
\newcommand{\argmin}{\operatorname*{arg\,min}}
\newcommand{\argmax}{\operatorname*{arg\,max}}

\title{\bf Predictive Compression Dynamics:\\
A Methodological Framework for Computable Information-Motivated Modeling}

\author[1]{Mats Helander}
\author[1]{Jeeves}
\affil[1]{Independent Research}
\date{Dated: October 21, 2025}

\begin{document}
\maketitle

\begin{abstract}
\noindent
We present \emph{Predictive Compression Dynamics} (PCD), a methodological recipe for constructing \emph{computable}, local functionals $\widehat{\Phi}$ and driving dynamics by gradient flow $\dot x=-\nabla\widehat{\Phi}(x)$ with preregistered parameters. Two concrete instances are given: (i) a fixed-graph pair functional and (ii) a smooth compact-support kernel; both yield an attractive inverse-square two-body form (after calibration) and admit Lyapunov descent. These models serve as methodological demonstrations of computable, information-motivated dynamics. We make the MDL split $L_{\text{tot}}=L(M)+L(D\mid M)$ explicit, give a minimal coding scheme linking $-\widehat{\Phi}$ descent to achievable $\Delta L_{\text{tot}}$, address well-posedness (smooth-kernel variant), recommend robust integrators (BAOAB for Langevin), and provide a preregistration/model-card template and falsifiers for a chosen model instance. The goal is a reproducible toolbox for compression-driven dynamics across domains.
\end{abstract}

\section{Positioning and Commitments}
A disciplined workflow to construct and test \emph{computable} local functionals $\widehat{\Phi}$ whose gradients define dynamics, with explicit preregistration (domain, discretization, kernels, parameters) and sanity checks.

\section{Operational Domain and Notation}
We consider $N$ point agents with positions $x_i\in\R^3$ and positive weights $m_i$. Computations use finite precision: lattice spacing $a_{\text{grid}}$ and $b$ bits/axis, stated \emph{a priori}. A global calibration constant $G_{\text{eff}}>0$ maps dimensionless forces to physical units. \emph{Design choice:} we use the same $m_i$ in the interaction and as inertial mass; this incidentally yields accelerations independent of $m_i$ by construction.

\section{Model--Data Decomposition and Coding Link}
Following MDL, we split description length as
\begin{equation}
\label{eq:Ltot}
L_{\text{tot}} = L(M) + L(D\mid M),
\end{equation}
where $L(M)$ encodes modeled regularities and $L(D\mid M)$ encodes residuals given $M$. A decrease $\Delta L_{\text{tot}}<0$ corresponds to realized compression. PCD treats a computable, local $\widehat{\Phi}$ as a proxy for (the negative of) an achievable $\Delta L_{\text{tot}}$; thus $\dot x=-\nabla\widehat{\Phi}$ implements a descent in achievable codelength under the chosen surrogate. 
This identification is heuristic and intended only as a modelling analogy.

\paragraph{Minimal explicit coding scheme.}
Let $(i,j)$ range over a symmetric set of ``near'' pairs. A two-part code describes (i) a shared pairwise template per distance bin and (ii) residual offsets:
\begin{itemize}
\item Partition distances into bins $\{B_k\}$ with centers $r_k$; encode the histogram counts with a prefix-free code.
\item For each pair $(i,j)$ with $r_{ij}\in B_k$, encode a residual offset relative to a shared template; the expected residual codelength per pair is a decreasing function $\ell(r_{ij})$.
\end{itemize}
Then, for fixed binning overhead and under mild regularity,
\begin{equation}
\label{eq:Ltotsum}
L_{\text{tot}} \approx \text{const} + \sum_{(i,j)} \ell(r_{ij}).
\end{equation}
Choosing $\widehat{\Phi}\propto -\sum_{(i,j)} \ell(r_{ij})$ makes $-\nabla\widehat{\Phi}$ a proxy for the gradient of achievable compression. A smooth choice $\ell(r)\approx (r^2+a^2)^{-1/2}$ yields closed-form forces below.

\section{Information-Motivated Functionals}
\subsection{Fixed-graph functional (corrected notation)}
Let $E\subset\{(i,j):\,1\le i<j\le N\}$ be a symmetric, degree-bounded edge set. Define
\begin{equation}
\label{eq:phiE}
\widehat{\Phi}_E(x)= -\sum_{(i,j)\in E} \frac{m_im_j}{\sqrt{\lVert x_i-x_j\rVert^2+a^2}},\qquad a>0.
\end{equation}
The minus sign ensures attraction under descent; $a$ regularizes collisions. The force on $i$ is
\begin{align}
\label{eq:forceE}
F^{(E)}_i(x)
&= -\nabla_{x_i}\widehat{\Phi}_E(x)
= -\sum_{\substack{j:\\(i,j)\in E}}
m_im_j\,\frac{(x_i-x_j)}{\big(\lVert x_i-x_j\rVert^2+a^2\big)^{3/2}}.
\end{align}
Two-body, $a\to 0$, $(i,j)\in E$ gives the attractive inverse-square form:
\begin{equation}
\label{eq:twoBody}
F^{(E)}_i \to -\,m_im_j\,\frac{x_i-x_j}{\lVert x_i-x_j\rVert^3}.
\end{equation}

\subsection{Smooth-kernel functional (well-posedness)}
To avoid neighbor-set discontinuities, choose a compactly supported, $C^1$ radial kernel $K_\sigma:[0,\infty)\to\R_{\ge 0}$ with support $\subset[0,R\sigma]$. Define
\begin{equation}
\label{eq:phiK}
\widehat{\Phi}_K(x)= -\sum_{i<j} m_im_j\,K_\sigma\!\big(\lVert x_i-x_j\rVert\big),
\end{equation}
so $F^{(K)}_i(x)=-\nabla_{x_i}\widehat{\Phi}_K(x)$ is continuous and locally Lipschitz off collisions. If $K_\sigma(r)\sim (r^2+a^2)^{-1/2}$ near $r=0$, one recovers the regularized two-body form \eqref{eq:twoBody}.

\section{Dynamics and Integrators}
With $\dot x=-\nabla\widehat{\Phi}(x)$,
\begin{equation}
\frac{d}{dt}\widehat{\Phi}(x(t)) = -\lVert\nabla\widehat{\Phi}(x(t))\rVert^2 \le 0,
\end{equation}
so $\widehat{\Phi}$ is a Lyapunov function. We preregister all parameters $(a_{\text{grid}},b,a,\sigma,\Delta t,m_i,\gamma,T,\text{seeds})$.

\paragraph{Deterministic gradient flow.}
Explicit Euler:
\begin{equation}
x_i^{(t+\Delta t)} = x_i^{(t)} + \Delta t\,F_i(x^{(t)}),\quad F_i\in\{F^{(E)}_i,F^{(K)}_i\}.
\end{equation}
For stability, use adaptive $\Delta t$ or semi-implicit variants.

\paragraph{Underdamped Langevin (BAOAB recommended).}
\begin{equation}
m_i\ddot{x}_i = F_i(x)-\gamma\,\dot{x}_i + \xi_i(t),\quad
\langle \xi_i(t)\xi_j(t')\rangle=2\gamma k_B T\,\delta_{ij}\delta(t-t').
\end{equation}
We recommend the BAOAB integrator with reported weak/strong orders.

\section{Sanity Checks}
With $a\to 0$ and a single pair, \eqref{eq:twoBody} holds (after one calibration $G_{\text{eff}}$).  
For $r\gg a$,
\begin{align}
\frac{r}{(r^2+a^2)^{3/2}}
= \frac{1}{r^2}\left(1-\frac{3a^2}{2r^2}+O\!\left(\frac{a^4}{r^4}\right)\right),
\end{align}
so
\begin{equation}
\big\lVert F^{(E)}_i\big\rVert
= m_im_j\,\frac{r}{(r^2+a^2)^{3/2}}
\approx m_im_j\,\frac{1}{r^2}\left(1-\frac{3a^2}{2r^2}\right).
\end{equation}
These expansions serve purely as numerical consistency checks.

\section{Well-posedness}
For $a>0$ and bounded degree, $\widehat{\Phi}_E\in C^1(\R^{3N}\setminus\{x_i=x_j\})$ and $F^{(E)}$ is locally Lipschitz off collisions. For $C^1$ kernels with bounded $K'_\sigma$, $F^{(K)}$ is continuous and locally Lipschitz. Existence and uniqueness follow by Picard–Lindelöf on compact intervals. For dynamic $k$NN, forces are piecewise smooth; employ hysteresis or prefer the smooth kernel.

\section{Preregistered Model Card (example)}
\textbf{Domain.} $a_{\text{grid}}=\SI{10}{\micro m}$, $b=16$. \\
\textbf{Functional.} $\widehat{\Phi}_K$ with Wendland $C^2$ kernel ($\sigma=\SI{0.5}{mm}$); softening $a=\SI{50}{\micro m}$. \\
\textbf{Dynamics.} BAOAB underdamped with $(m_i\equiv 1,\gamma=0.1,T=\SI{300}{K})$, $\Delta t=\SI{1e-3}{s}$. \\
\textbf{Calibration.} Single $G_{\text{eff}}$ fit in a dilute two-body sandbox at $r\gg a$. \\
\textbf{Sanity checks.} Verify \eqref{eq:twoBody} and the far-field expansion; report seeds and residuals.

\section{Falsifiers for a chosen instance}
Given fixed $(\widehat{\Phi},\text{params})$, declare the instance falsified if:
\begin{enumerate}[label=(F\arabic*)]
\item Two-body trajectories disagree with the calibrated inverse-square form beyond numerical error.
\item Smooth-kernel vs fixed-graph variants differ systematically at small $r$ beyond topology effects.
\item The code-length proxy correlates poorly with realized compression in controlled tests (e.g.\ $-\widehat{\Phi}$ vs measured $L_{\text{tot}}$).
\end{enumerate}

\section{Discussion and Scope}
PCD supplies a reproducible route from \emph{computable} information-motivated functionals to concrete dynamics. Future work will extend this framework to broader estimator families under the same preregistration discipline.

\section*{Acknowledgments}
We thank colleagues for discussions on local estimators, kernels, integrators, and $N$-body numerics. Earlier drafts that explored alternative framings are superseded by this methodological formulation.

\section*{References}
\begin{thebibliography}{12}
\bibitem{Shannon1948}
C.\ E.\ Shannon, ``A mathematical theory of communication,'' \emph{Bell Syst.\ Tech.\ J.} (1948).
\bibitem{Rissanen1978}
J.\ Rissanen, ``Modeling by shortest data description,'' \emph{Automatica} (1978).
\bibitem{Levin}
L.\ A.\ Levin, ``On the notion of a random sequence,'' \emph{Sov.\ Math.\ Dokl.} (1971).
\bibitem{Amari2016}
S.\ Amari, \emph{Information Geometry and Its Applications}, Springer (2016).
\bibitem{JordanKinderlehrerOtto1998}
R.\ Jordan, D.\ Kinderlehrer, F.\ Otto, ``The variational formulation of the Fokker–Planck equation,'' \emph{SIAM J.\ Math.\ Anal.} \textbf{29} (1998).
\bibitem{Caticha2012}
A.\ Caticha, \emph{Entropic Inference and the Foundations of Physics}, (2012).
\bibitem{Wendland1995}
H.\ Wendland, ``Piecewise polynomial, positive definite and compactly supported radial functions,'' \emph{Adv.\ Comput.\ Math.} \textbf{4} (1995) 389--396.
\bibitem{LeimkuhlerMatthews2013}
B.\ Leimkuhler, M.\ Matthews, ``Rational construction of stochastic numerical methods for molecular sampling,'' \emph{Appl.\ Math.\ Res.\ eXpress} (2013).
\bibitem{LeimkuhlerMatthews2016}
B.\ Leimkuhler, C.\ Matthews, \emph{Molecular Dynamics}, Springer (2016).
\bibitem{Prim1957}
R.\ C.\ Prim, ``Shortest connection networks and some generalizations,'' \emph{Bell Syst.\ Tech.\ J.} \textbf{36}, 1389--1401 (1957).
\bibitem{Hernquist1990}
L.\ Hernquist, ``An analytical model for spherical galaxies and bulges,'' \emph{ApJ} \textbf{356}, 359--364 (1990).
\bibitem{Plummer1911}
H.\ C.\ Plummer, ``On the problem of distribution in globular star clusters,'' \emph{MNRAS} \textbf{71}, 460--470 (1911).
\end{thebibliography}

\end{document}

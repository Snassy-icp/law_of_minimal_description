\documentclass[11pt,a4paper]{article}

\usepackage[utf8]{inputenc}
\usepackage[T1]{fontenc}
\usepackage{lmodern}
\usepackage{geometry}
\geometry{margin=1in}
\usepackage{microtype}
\usepackage{amsmath,amssymb,amsfonts,amsthm}
\usepackage{mathtools}
\usepackage{physics}
\usepackage{siunitx}
\usepackage{graphicx}
\usepackage{booktabs}
\usepackage{hyperref}
\usepackage[nameinlink]{cleveref}
\usepackage{authblk}
\usepackage{enumitem}
\usepackage{tikz}
\setlist{nosep}

\hypersetup{
  colorlinks=true,
  linkcolor=blue!60!black,
  citecolor=blue!60!black,
  urlcolor=blue!60!black
}

\numberwithin{equation}{section}

% Macros
\newcommand{\R}{\mathbb{R}}
\newcommand{\EE}{\mathbb{E}}
\newcommand{\KL}{\operatorname{KL}}
\newcommand{\argmin}{\operatorname*{arg\,min}}
\newcommand{\argmax}{\operatorname*{arg\,max}}

\title{\bf Predictive Compression Dynamics:\\
A Methodological Framework for Computable Information-Driven Modeling}

\author[1]{Mats Helander}
\author[1]{Jeeves}
\affil[1]{Independent Research}
\date{Dated: October 21, 2025}

\begin{document}
\maketitle

\begin{abstract}
\noindent
We present \emph{Predictive Compression Dynamics} (PCD) as a \emph{methodological framework} for building \emph{computable} local functionals that drive dynamics by explicit gradient flow. Rather than postulating uncomputable principles, PCD starts from an \emph{information-motivated} surrogate codelength $\widehat{\Phi}$ and specifies dynamics $\dot{x}\propto -\nabla\widehat{\Phi}(x)$ with all parameters preregistered. We study two concrete classes: (i) a \emph{fixed-graph} functional
\[
\widehat{\Phi}_E(x)= -\!\!\sum_{(i,j)\in E}\frac{m_im_j}{\sqrt{\lVert x_i-x_j\rVert^2+a^2}},
\]
and (ii) a \emph{smooth kernel} functional
\[
\widehat{\Phi}_K(x)= -\sum_{i<j} m_im_j\,K_\sigma(\lVert x_i-x_j\rVert),
\]
with $K_\sigma$ compactly supported and $C^1$. The first provides a transparent sandbox; the second avoids neighbor-set discontinuities and guarantees well-posedness. In the dilute two-body limit both produce an attractive $1/r^2$ \emph{form} (after calibration), while the softening $a$ regularizes collisions. These are \emph{operational, information-motivated models}, not new claims about gravity; familiar “Plummer-like” roll-offs serve here as \emph{sanity checks}, not phenomenological predictions. We give preregistered algorithms, an explicit model--data codelength decomposition, and clear falsification criteria for model instances. The framework supports reproducible exploration of \emph{computable} compression-driven dynamics across domains.
\end{abstract}

\section{Positioning and Commitments}
\textbf{What this paper is:} a disciplined way to construct and test \emph{computable} local functionals $\widehat{\Phi}$ whose gradients define dynamics. We make code-level, preregistered choices explicit (domains, discretizations, kernels, parameters) and provide sanity checks.

\medskip
\noindent\textbf{What this paper is \emph{not}}: a claim of novel gravitational phenomenology, nor a derivation of GR/QM. Any resemblance of simple pair terms to familiar softened potentials is acknowledged and used only as a controlled testbed.

\section{Operational Domain and Notation}
We consider $N$ point agents with positions $x_i\in\R^3$ and positive weights $m_i$ (``masses''). Computations occur at finite precision: lattice spacing $a_{\text{grid}}$ and $b$ bits per coordinate stated \emph{a priori}. A global calibration constant $G_{\text{eff}}>0$ turns the dimensionless force into physical units.

\section{Information-Motivated Functionals}

\paragraph{Model--data decomposition (explicit).}
Following standard MDL reasoning, a description length splits as
\begin{equation}
\label{eq:Ltot}
L_{\text{tot}} = L(M) + L(D\mid M),
\end{equation}
where $L(M)$ encodes the model/regularities and $L(D\mid M)$ the residual data given that model. A decrease $\Delta L_{\text{tot}}<0$ corresponds to \emph{realized compression}. In PCD we treat a computable, local $\widehat{\Phi}$ as a proxy for the \emph{negative} of an achievable $\Delta L_{\text{tot}}$; gradient descent in $\widehat{\Phi}$ heuristically implements compression.

\subsection{Fixed-graph functional}
Let $E\subset\{(i,j):\,1\le i<j\le N\}$ be a symmetric, degree-bounded edge set (e.g.\ an initial $k$NN graph). Define
\begin{equation}
\label{eq:phiE}
\widehat{\Phi}_E(x)= -\sum_{(i,j)\in E} \frac{m_im_j}{\sqrt{\lVert x_i-x_j\rVert^2+a^2}},
\qquad a>0.
\end{equation}
\emph{Sign convention:} the minus sign ensures attraction under gradient descent. Softening $a$ regularizes collisions. The force on $i$ is
\begin{align}
\label{eq:forceE}
F^{(E)}_i(x) 
&= -\nabla_{x_i}\widehat{\Phi}_E(x) \nonumber\\
&= -\sum_{\substack{j:\\(i,j)\in E}} 
m_im_j\,\frac{(x_i-x_j)}{\big(\lVert x_i-x_j\rVert^2+a^2\big)^{3/2}}.
\end{align}
In the two-body, $a\to0$ limit with $(i,j)\in E$,
\begin{equation}
\label{eq:twoBody}
F^{(E)}_i \to -\,m_im_j\,\frac{x_i-x_j}{\lVert x_i-x_j\rVert^3},
\end{equation}
i.e.\ an attractive $1/r^2$ \emph{form} along the active edge. For physical units we use $F^{\text{phys}}_i=G_{\text{eff}}\,F^{(E)}_i$.
\paragraph{Remark (edge modeling).} Forces vanish between pairs not in $E$; this is a \emph{modeling choice} for locality/sparsity, not a universal law.

\subsection{Smooth-kernel functional}
To avoid kNN neighbor-set discontinuities, choose a compactly supported, $C^1$ radial kernel $K_\sigma:[0,\infty)\to\R_{\ge 0}$ with support $[0,R\sigma]$ (e.g.\ a Wendland $C^2$ kernel). Define
\begin{equation}
\label{eq:phiK}
\widehat{\Phi}_K(x)= -\sum_{i<j} m_im_j\,K_\sigma\!\big(\lVert x_i-x_j\rVert\big).
\end{equation}
Then $F^{(K)}_i(x)=-\nabla_{x_i}\widehat{\Phi}_K(x)$ is continuous and locally Lipschitz (away from collisions if $K_\sigma'(0)$ finite). For $K_\sigma(r)\sim (r^2+a^2)^{-1/2}$ near zero one recovers the same regularized two-body \emph{form} as \eqref{eq:twoBody}.

\subsection{A minimal codelength view}
Interpreting $-\widehat{\Phi}$ as a surrogate \emph{gain in description length} when nearby pairs are represented jointly, if the expected residual codelength per pair scales like a decreasing $\ell(r_{ij})$, then
\[
L_{\text{tot}} \approx \text{const} + \sum_{(i,j)} \ell(r_{ij}),
\quad\Rightarrow\quad
\widehat{\Phi}\;\propto\; -\sum_{(i,j)} \ell(r_{ij}).
\]
We refrain from claiming exact MDL optimality; $\widehat{\Phi}$ is an \emph{operational proxy} to be tested.

\section{Dynamics and Preregistration}
All parameters $(a_{\text{grid}},b,a,\sigma, \Delta t, m_i,\gamma,T,\text{seeds})$ are fixed \emph{before} experiments.

\paragraph{Deterministic gradient flow.}
\begin{equation}
x_i^{(t+\Delta t)}=x_i^{(t)}+\Delta t\,F_i(x^{(t)}),\quad 
F_i\in\{F^{(E)}_i,F^{(K)}_i\}.
\end{equation}

\paragraph{Underdamped Langevin.}
\begin{equation}
m_i\ddot{x}_i = F_i(x)-\gamma\,\dot{x}_i + \xi_i(t),\quad
\langle \xi_i(t)\xi_j(t')\rangle=2\gamma k_B T\,\delta_{ij}\delta(t-t').
\end{equation}

\section{Sanity Checks (not phenomenology)}
\textbf{Two-body limit.} With $a\to 0$ and a single interacting pair, \eqref{eq:twoBody} holds (after calibration $G_{\text{eff}}$).\\
\textbf{Softening expansion (corrected).} For $r=\lVert x_i-x_j\rVert\gg a$,
\begin{align}
\label{eq:softSeries}
\frac{r}{(r^2+a^2)^{3/2}}
= \frac{1}{r^2}\left(1-\frac{3a^2}{2r^2}+O\!\left(\frac{a^4}{r^4}\right)\right),
\end{align}
hence
\begin{equation}
\label{eq:forceCorrect}
\big\lVert F^{(E)}_i\big\rVert
= m_im_j\,\frac{r}{(r^2+a^2)^{3/2}}
\;\stackrel{r\gg a}{=}\; m_im_j\,\frac{1}{r^2}\Big(1-\frac{3a^2}{2r^2}+\dots\Big).
\end{equation}
We use \eqref{eq:softSeries} solely to verify numerics; \emph{no claim} is made that nature exhibits such a roll-off at any laboratory scale.

\section{Well-posedness}
For $a>0$ and bounded degree, $\widehat{\Phi}_E\in C^1(\R^{3N}\setminus\{x_i=x_j\})$ and $F^{(E)}$ is locally Lipschitz away from collisions; similarly $F^{(K)}$ is continuous and locally Lipschitz for $C^1$ kernels with bounded derivative at $0$. If $k$NN is used with dynamic neighbors, forces are only piecewise smooth; then employ (i) hysteresis for neighbor swaps, (ii) temporal smoothing of weights, or (iii) prefer the smooth kernel model \eqref{eq:phiK}.

\section{A Preregistered Model Card (example)}
\textbf{Domain.} $a_{\text{grid}}=\SI{10}{\micro m}$, $b=16$. \\
\textbf{Functional.} $\widehat{\Phi}_K$ with Wendland $C^2$ kernel of scale $\sigma=\SI{0.5}{mm}$; softening $a=\SI{50}{\micro m}$. \\
\textbf{Dynamics.} Underdamped with $(m_i\equiv 1,\gamma=0.1,T=\SI{300}{K})$, $\Delta t=\SI{1e-3}{s}$. \\
\textbf{Calibration.} $G_{\text{eff}}$ fit once in a dilute two-body sandbox at $r\gg a$. \\
\textbf{Sanity checks.} Verify \eqref{eq:twoBody} form and \eqref{eq:softSeries} numerically; report seeds and residuals.

\section{What to Falsify (for a given instance)}
Given fixed $(\widehat{\Phi},\text{params})$, the instance is falsified if:
\begin{enumerate}[label=(F\arabic*)]
\item Two-body trajectories disagree with the calibrated $1/r^2$ \emph{form} beyond stated error under identical numerical conditions.
\item Kernel vs.\ fixed-graph variants yield statistically inconsistent small-$r$ behavior not explainable by topology (violates within-class robustness).
\item The chosen code-length proxy (e.g.\ $-\widehat{\Phi}$ vs.\ an external structural complexity metric) fails to correlate in controlled tests.
\end{enumerate}
These are \emph{model-level} falsifiers; they do not purport to test fundamental physics.

\section{Discussion and Scope}
PCD supplies a reproducible route from \emph{computable} information-motivated functionals to concrete dynamics. Simple pairwise terms coincide with well-known softened interactions; this is a feature for validation, not a novelty claim. The framework can be extended to richer $\widehat{\Phi}$ (e.g.\ graph Laplacians, learned local codes) with the same preregistration discipline.

\section*{Acknowledgments}
We thank colleagues for discussions on local estimators, kernels, and $N$-body numerics. Any prior drafts suggesting phenomenology are superseded by this methodological formulation.

\section*{References}
\begin{thebibliography}{9}
\bibitem{Shannon1948}
C.~E.~Shannon, ``A mathematical theory of communication,'' \emph{Bell Syst.\ Tech.\ J.} (1948).
\bibitem{Rissanen1978}
J.~Rissanen, ``Modeling by shortest data description,'' \emph{Automatica} (1978).
\bibitem{Plummer1911}
H.~C.~Plummer, ``On the problem of distribution in globular star clusters,'' \emph{MNRAS} \textbf{71}, 460--470 (1911).
\bibitem{Wendland}
H.~Wendland, ``Piecewise polynomial, positive definite and compactly supported radial functions,'' \emph{Adv.\ Comput.\ Math.} \textbf{4} (1995) 389--396.
\end{thebibliography}

\appendix

\section{Schematic Field (single edge)}
\begin{figure}[h]
\centering
\begin{tikzpicture}[>=stealth]
  % Particles
  \fill (-2,0) circle (2pt) node[below] {$i$};
  \fill ( 2,0) circle (2pt) node[below] {$j$};
  % Left-side arrows pointing right
  \draw[->] (-3.2, 0.8) -- (-2.6, 0.8);
  \draw[->] (-3.2, 0.4) -- (-2.7, 0.4);
  \draw[->] (-3.2, 0.0) -- (-2.8, 0.0);
  \draw[->] (-3.2,-0.4) -- (-2.7,-0.4);
  \draw[->] (-3.2,-0.8) -- (-2.6,-0.8);
  % Right-side arrows pointing left
  \draw[->] ( 3.2, 0.8) -- ( 2.6, 0.8);
  \draw[->] ( 3.2, 0.4) -- ( 2.7, 0.4);
  \draw[->] ( 3.2, 0.0) -- ( 2.8, 0.0);
  \draw[->] ( 3.2,-0.4) -- ( 2.7,-0.4);
  \draw[->] ( 3.2,-0.8) -- ( 2.6,-0.8);
\end{tikzpicture}
\caption{Attractive field along the active edge for \eqref{eq:phiE}; magnitude regularized near particles by $a$.}
\end{figure}

\section{Units and the corrected far-field}
With $x$ in length units, \eqref{eq:phiE} has units of inverse length. The calibrated constant $G_{\text{eff}}$ supplies force units so that $F^{\text{phys}}=G_{\text{eff}}F$. The large-$r$ expansion used for numerical validation is \eqref{eq:softSeries} and \eqref{eq:forceCorrect}.
\end{document}
